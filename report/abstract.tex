\thispagestyle{plain}
\begin{center}
    \Large
    \textbf{Abstract}
\end{center}
In this first half of the project, I learned the basics of quantum computing and information followed by the study of quantum noise and operations, the book \ref{book:nielsen_chuang} has helped me a lot in this process and I've done most of my study from this book. Starting with linear algebra, I grasped new "quantum mechanical" notation followed by several theorems and definitions relevant to quantum mechanics like commutators, anticommutators, tensor products and polar and singular value decompositions etc. I then  built a basic foundation of quantum mechanics by learning its postulates along with density operators. Applications like superdense coding helped in gaining more insight. This was followed by an overview of quantum information and computing by studying about qubits, quantum gates, quantum circuits, the no-cloning theorem, quantum teleportation etc. I studied few quantum algorithms like Deutsch's algorithm, Deutsch-Jozsa algorithm followed by Stern-Gerlach experiment which provided an experimental overview of quantum information processing. I then studied about quantum noise and quantum operations where I first got to properly understand what "noise" means, atleast in classical sense.I then studied different approaches to understand quantum operations, most importantly operator-sum representation. This was followed by understanding few examples of quantum noise and operations using Bloch sphere, generated by bit-flip, phase-flip, bit-phase-flip, depolarizing channels. At the end, I've covered amplitude and phase damping. I wrote this report simultaneouly while I was studying, please forgive any typos.